\chapter{Yhteenveto\label{ch:yhteenveto}}

Tutkielmassa käsiteltiin salasanojen tiivisteiden murtamista sekä siihen käytettyjä menetelmiä. Havaittiin, että salasanojen tiivisteitä voidaan murtaa varsin tehokkaasti. Jos käytössä oleva laskentatehon määrä jatkaa kasvuaan Mooren lain mukaisesti, tulee salasanojen jatkossa olla yhä pidempiä ollakseen turvallisia.

Toinen keskeinen tutkimuskysymys liittyi murtamisen menetelmiltä suojautumiseen. Havaittiin, että hyökkäyksiltä suojautuminen on itse asiassa varsin monimutkaista. Ohjelmistokehittäjän on osattava ottaa huomioon erilaiset hyökkäysmenetelmät ja käyttäjän on ymmärrettävä vahvan salasanan piirteet sekä salasanojensa oikeaoppinen säilyttäminen. Puutteet salasanojen säilytyksessä tai huonosti valittu salasana mahdollistavat tiivisteiden nopeamman murtumisen.

Salasanojen tietoturva on monimutkainen, mutta myös kriittinen kysymys. Koska salasanat ovat edelleen yleisin todentamiseen käytetty menetelmä, on aihepiirin tutkiminen myös jatkossa erityisen tärkeää. Aihe koskettaa käytännössä jokaista Internetin käyttäjää. Etenkin suojautumiseen on kiinnitettävä huomiota, sillä tiivisteiden murtaminen ei ole teknisesti edes kovin haastavaa. Vuotaneita salasanoja ja niiden tiivisteitä on miljoonittain saatavilla ja ohjelmistot kuten Hashcat mahdollistavat hyökkäyksen suorittamisen omalla tietokoneella jopa ilman erityistä ohjelmointitaitoa.

Tutkielman keskeinen havainto on, että salasanojen tiivisteiden murtaminen on mahdollista lähinnä siksi, että ihmiset valitsevat heikkoja salasanoja. Uniikkeja, riittävän pitkiä ja satunnaisesti generoituja salasanoja ei pystytä nykyisillä menetelmillä murtamaan kohtuullisessa ajassa. Jos jokainen käyttäjä valitsisi riittävän vahvan salasanan eikä myöskään joutuisi sosiaalisen manipuloinnin tai haittaohjelmien uhriksi, eivät hyökkääjät saisi ensimmäistäkään salasanaa murrettua. Tämä tosin lienee toiveajattelua ja käytännössä mahdotonta.

Voidaan myös argumentoida, että salasanatietokannan vuotaminen ulkopuolisille ja murtamisen mahdollistaminen on sovelluskehittäjän syytä. Tämän takia myös kehittäjällä on vastuu salasanojen turvallisesta säilyttämisestä, sopivan tiivistefunktion valitsemista ja kryptografisen suolan käytöstä. Sovelluskehittäjän vastuulla on myös sopivien salasanakäytäntöjen määritteleminen sekä tuen lisääminen monivaiheiselle tunnistautumiselle.

Salasanojen ilmeisistä ongelmista johtuen myös vaihtoehtoisia keinoja käyttäjän todentamiseen on syytä kehittää. Erilaisia ratkaisuja kuten biometrinen tunnistus tai kertakäyttöiset salasanat on jo laajalti käytössä. Verkkopankkeihin kirjaudutaan usein erillisillä pankin myöntämillä tunnuksilla sekä erillisellä tunnuslukulistalla tai sovelluksella sen sijaan, että pankin asiakas päättäisi itse käyttäjätunnuksensa ja salasanansa.

Yhteenvetona voidaan todeta, että salasanojen tiivisteihin liittyviä ongelmia voidaan ehkäistä tiedon avulla. Asiaan perehtyneet sovelluskehittäjät luovat turvallisempia sovelluksia ja valveutuneet käyttäjät valitsevat parempia salasanoja.
